\documentclass[	pdftex, 
								a4paper,
								11pt, DIV11, BCOR5mm,
								parskip,
								%openany,
								]{scrreprt}
\usepackage[utf8]{inputenc}
\usepackage[T1]{fontenc}

\usepackage[colorlinks=true,linkcolor=blue,citecolor=blue]{hyperref}
\usepackage{comment}
\usepackage[ngerman]{babel}
\usepackage[nonumberlist,toc,acronym]{glossaries}

\usepackage{fancyhdr}
\usepackage[section]{placeins}
\usepackage{graphicx}
\usepackage{listings}
\usepackage[german]{babelbib}
\usepackage{array}
\usepackage{wrapfig}
\usepackage{varwidth}
\usepackage{float}
\usepackage{dirtree}
\usepackage{epstopdf}
\usepackage{textpos}
\usepackage{changepage}
\usepackage{theorem}
\usepackage{caption}
\usepackage{framed}
\usepackage{booktabs}
\usepackage{tikz}
\usetikzlibrary{decorations.markings,arrows,positioning,trees,calc,fit,shapes}
\usepackage[noend]{algorithmic}
\usepackage{algorithm}
\usepackage{environ}
\usepackage{ulem}

%Kopfzeile hinzufuegen
\renewcommand{\sectionmark}[1]{\markboth{#1}{}} % set the \leftmark

\fancypagestyle{plain}{
    
    \fancyhead[R]{\leftmark}
    \fancyhead[L]{\myauthor}
    \renewcommand{\headrulewidth}{1pt}
    %\fancyfoot[R]{\thepage}
}

\pagestyle{fancy}

%Literaturverzeichnis hinzufuegen
\bibliographystyle{babplain}

% Glossar hinzufuegen
\makeglossaries
\newacronym{usda} {USDA}{US Department of Agriculture}
\newacronym{fdc}{FDC}{FoodData Central}
\newacronym{rest}{REST}{Representational state transfer}
\newacronym{dge}{DGE}{Deutsche Gesellschaft für Ernährung}
\newacronym{pal}{PAL-Value}{physical activity level-Value}
\newacronym{json}{JSON}{JavaScript Object Notation}
\newacronym{iri}{IRI}{International Resource Identifier}
\newacronym{http}{HTTP}{HyperText Transfer Protocol}
\newacronym{xml}{XML}{Extensible Markup Language}
\newacronym{cdno}{CDNO}{Compositional Dietary Nutrition Ontology}
\newacronym{obo}{OBO}{Open Biological and Biomedical Ontology}
\newacronym{sparql}{SPARQL}{SPARQL Protocol And RDF Query Language}
\newacronym{sql}{SQL}{Structured Query Language}
\newacronym{uri}{URI}{Uniform Resource Identifier}
\newacronym{gi}{GI}{Glykämischer Index}
\newacronym{bmi}{BMI}{Body-Mass-Index}



\begin{document}
\setcounter{page}{-2}
\pagestyle{empty}
\pagenumbering{none}
\newcommand{\firstreviewer}{Prof. Michael Beetz PhD}
\newcommand{\secondreviewer}{}
\newcommand{\supervisor}{Michaela Kümpel}
\newcommand{\thesistype}{Master Thesis}
\newcommand{\myauthor}{Naser Azizi, Sorin Arion}
\newcommand{\mymaintitle}{A Knowledgebase with which you can generate robot plan for multiple mixing actions}
\newcommand{\mysubtitle}{A nutrition ontology with web interface for customer specific dietary \newline recommendations} 
\newcommand{\mytitle}{\centering {\Huge \mymaintitle}\\[.3in] 
    {\Large \mysubtitle}}
\newcommand{\pdftitle}{\mymaintitle - \mysubtitle}
\newcommand{\formattedfronttitle}{{\mytitle}}
\newcommand{\formattedinnertitle}{{\mytitle}}

\begin{titlepage}
	\vspace*{-2.2cm}
	\begin{adjustwidth}{-0cm}{-2.3cm}
	\thispagestyle{empty}
        \begin{figure}
            \begin{minipage}{.4\linewidth}
	\begin{flushleft}
		\includegraphics[height=1.5cm]{Graphics/unilogo-transp.pdf}
	\end{flushleft}
    \end{minipage}
    \hspace{.2\linewidth}
            \begin{minipage}{.4\linewidth}
	\begin{flushright}
		\includegraphics[height=3.0cm]{Graphics/logo-ai-small.pdf}
	\end{flushright}
    \end{minipage}
\end{figure}
	  \vfill
	  %\scalebox{0.95}{
    	  %\begin{minipage}{1.2\textwidth}
    	  %{
    		%\formattedfronttitle 
    	  %}
    	  %\end{minipage}
	  %}
	\begin{center}
	  {\huge \mymaintitle}\\
	  
	  \vfill
	  {\Large \thesistype}\\[2.5ex]
	  {\Large\em \myauthor}
	  \vfill
	{  
      \renewcommand\arraystretch{1.5}
      \begin{tabular}{l@{\hspace{2em}}r@{\hspace{1ex}}p{7cm}}
     Pr\"ufer der \thesistype: & 1. & \firstreviewer\\
                                 & 2. & \secondreviewer\\
	Supervisor		    &   & \supervisor\\
   \end{tabular}
  }
	\end{center}
	\end{adjustwidth}
\end{titlepage}
	\pagenumbering{arabic}
	\setcounter{page}{1}
	\pagestyle{plain}
	%\addcontentsline{toc}{chapter}{Eidesstaatliche Erkl\"arung}
	\chapter*{Eidesstattliche Erkl\"arung}
	
	Hiermit erkl\"aren wir, dass die vorliegende Arbeit selbstst\"andig angefertigt,
	nicht anderweitig zu Pr\"ufungszwecken vorgelegt und keine anderen als die
	angegebenen Hilfsmittel verwendet habe. S\"amtliche wissentlich verwendete
	Textausschnitte, Zitate oder Inhalte anderer Verfasser wurden ausdr\"ucklich als
	solche gekennzeichnet.
	
	Bremen, den \makeatletter\@date\makeatother
	
	\vspace*{1em}
	\rule{15em}{0.16667pt}\\
	\author{Naser Azizi, Sorin Arion}
	\makeatletter\@author\makeatother
	
	
	\normalem
	%Abstrakt
	%\addcontentsline{toc}{chapter}{Abstract}


	\chapter*{Motivation}
	
	Although industrial robots have been considered standard for many years, the widespread adoption of AI-based autonomous household robots is still far from being anticipated in every household.
	Obstacles such as unfamiliar surroundings and a limited knowledge base can significantly hinder the ability of an autonomous robot to effectively respond in a given situation, hence, the robot must precisely plan for every minor detail, even those that might appear trivial to us humans.
	To achieve the defined objectives, a proficient and comprehensive robotic system including Control, Perception, Navigation, and Knowledge is necessary.
	An illustrative goal could be cooking. To execute this task, the robot must possess various capabilities, including placing items, grasping objects, pouring liquids, cutting ingredients, and mixing components.	Those things seem trivial to us humans, but for the robots it requires a lot of Implementation and information.
	In this thesis, we aim to introduce an approach detailing how a robotic system can execute mixing actions effectively.	
	The challenges arise from the diverse methods of mixing, which further vary based on the specific ingredients being combined. Additionally, the consideration of compatible Mixing Tools and Mixing Containers is crucial, as not every tool can be utilized with every container.
	Through the incorporation of a knowledge graph containing rules related to to various actions, we aim to empower the robotic system to make informed decisions on the appropriate motions to employ. This decision-making process will take into account the specific task at hand and the involved ingredients.
	By incorporating this knowledge graph, we aim to advance towards autonomous robots capable of engaging in cooking activities.

	\chapter*{Introduction}
	First, we intend to analyze existing research, particularly examining studies that delve into diverse mixing motions and exploring related systems that contain a queryable knowledge graph.
	In the following section, we will introduce the reader with the frameworks employed to accomplish our objectives.	
	Subsequently, we will present our approach to data acquisition and data representation.
	In this chapter, we aim to explain the mixing tasks under consideration and articulate our methodology for representing this data to enable effective querying.
	In the subsequent section, we will illustrate the implementation of our rules, with the ultimate aim of determining the appropriate motion to be employed.
	To validate our concept, we have opted to simulate various scenarios and assess their outcomes, demonstrating the efficacy of our implemented system.
	Finally, we will delve into the discussion of our results and draw conclusions to wrap up our work.
	
\chapter{Related Work}
\label{chap:Related_work}

In diesem Kapitel möchten wir auf die verschiedenen Arbeiten hinweisen, die ähnlich zu unserer sind.
Da unsere Arbeit sich auf zwei Hauptkomponente bezieht, nämlich der Mixing Motions und der dementsprechenden Wissensrepräsentation, werden wir dieses Kapitel in zwei Abschnitten unterteilen, die sich jeweils mit dem Fokus befassen.
\section*{Knowledge Representation}
In diesem Abschnitt untersuchen wir ähnliche Arbeiten, die sich mit der Wissensrepräsentation befassen, welche abfragbare Parameter und Entscheidungen enthält, welche dazu führen sollen, die richtige Motion auszuführen.
\subsection*{FOON}
\subsection*{Cutting}
FoodCutting aims to equip robotic agents with necessary information about how to 
execute  cutting tasks in unknown environments for the household domain. From high level plans FoodCutting breaks down 
the cutting task, part of some robot plan, into executable motions regarding cutting fruits and vegetables. These motions are parameterized 
by some technique and repetitions to achieve the agents objective. 
FoodCutting does not require fully available knowledge
about the agents environment, instead the robot should be capable of recognizing certain objects for cutting 
operations. 
\subsection*{Pouring?}
\section*{Motions}
Dieser Abschnitt stellt die Arbeiten vor, welche sich mit der Motion Plannung von verschiedenen Mixing Bewegungen auseinadersetzt.
\subsection*{BakingBot}
BakeBot realised on the PR2 robotics platform attempts to achieve baking cookies. An implementation of locating relevant things for MixingTasks
like a bowl and ingredients has been realised for semi-structured environments. An algorithm to perform mixing motions has been implemented as well, to mix 
ingredients with different characteristics into an uniform dough. These motions are limited to a simple circular and linear mixing motion.
The authors follow an bottom-up approach which is inherently motion driven rather than a
symbolic one which attempts to break down high level tasks into executable low level motions. 

\subsection*{Mixing Paper}
\subsection*{FluidLab}
FluidLab is a simulation environment for different kinds of manipulation tasks regarding liquids. Its underlying engine uses differentiable physics 
enabling reinforcement learning and optimization techniques in manipulation to utilize the engine, to achieve several tasks 
including liquids and solids, like mixing tasks. 

\newpage
\begin{itemize}
    \item \url{https://ieeexplore.ieee.org/stamp/stamp.jsp?tp=&arnumber=1641754}: relativ alte Quelle, dort möchte man einen Hilfsroboter vorstellen, der mit dem Menschen zusmamen Teilaufgaben im Bereich des Kochens übernehmen kann. Mixing wird als special motion erwähnt und es werden keine Beispiele einer Motionausführung im Bereich des Mixings angeboten. Quelle weiter benutzen?: Eher nicht
    \item \url{https://ieeexplore.ieee.org/stamp/stamp.jsp?tp=&arnumber=8954776}: Bezieht sich kaum auf Mixing Aktion, wird erwähnt dass es getan wird, jedoch nicht weiter darauf eingegangen, reinnehmen? Nein.
    \item \url{https://robomechjournal.springeropen.com/articles/10.1186/s40648-021-00204-6}
    \item \url{https://www.researchgate.net/profile/Daniela_Rus/publication/265243176_BakeBot_Baking_Cookies_with_the_PR2/links/56d043ad08aeb52500cd34a0.pdf}
    \item \url{https://ieeexplore.ieee.org/stamp/stamp.jsp?tp=&arnumber=9083695}: Mixing steht drin, allerdings nur high level commands sowas wie: Mix all the ingredients with a blender, es wird nicht gesagt wie man das mixen soll oder womit etc. Reinnehmen?: Vlt als Beispiel für MixingAktionen wo nicht klar wird wie das abgebildet wird auf motions etc.
    \item \url{https://ieeexplore.ieee.org/stamp/stamp.jsp?tp=&arnumber=7301404}
    \item \url{https://ieeexplore.ieee.org/abstract/document/8460964}
    \item \url{https://ieeexplore.ieee.org/stamp/stamp.jsp?tp=&arnumber=9096241}: Paper, dass die Cooking Roboter beschreibt und kategorisiert, gut als Anlaufstelle um Quellen zu erhalten. Reinnehmen?: Eventuell wenn wir irgendwie kategorisieren wollen wo unser system steht.
    \item \url{https://ieeexplore.ieee.org/stamp/stamp.jsp?tp=&arnumber=10004056}
    \item \url{https://robbreport.com/gear/electronics/moley-robotics-robot-kitchen-uk-for-sale-1234590791/}
    \item \url{https://ieeexplore.ieee.org/stamp/stamp.jsp?tp=&arnumber=8310925}
    \item \url{https://ieeexplore.ieee.org/stamp/stamp.jsp?tp=&arnumber=7523919}: Nein
    \item \url{https://ieeexplore.ieee.org/stamp/stamp.jsp?tp=&arnumber=7523919}
    \item \url{https://www.researchgate.net/publication/361723447_The_use_of_Robotics_in_the_Kitchens_of_the_Future_The_example_of_Moley_Robotics/link/63beb455a99551743e5d609a/download?_tp=eyJjb250ZXh0Ijp7ImZpcnN0UGFnZSI6InB1YmxpY2F0aW9uIiwicGFnZSI6InB1YmxpY2F0aW9uIn19}
    \item \url{https://ieeexplore.ieee.org/stamp/stamp.jsp?tp=&arnumber=6100855}
    \item \url{https://cs.brown.edu/people/stellex/publications/bollini12.pdf} BakeBot Paper related, Mixing wird kurz erwähnt, welche Bewegungen gemacht werden, jedoch nicht in abhängigkeit von irgendwelchen Gegebenheiten wie zb Zutaten. Reinnehmen?: Ja, wenn es sonst nichts gibt ...
\end{itemize}

There are multiple approaches for mixing tasks in the kitchen domain. From high level symbolic to low level motions and 
learning approaches, these different approaches, tries to tackle the challenge of solving mixing in the kitchen environment.

FoodCutting aims to equip robotic agents with necessary information about how to 
execute  cutting tasks in unknown environments for the household domain. From high level plans FoodCutting breaks down 
the cutting task, part of some robot plan, into executable motions regarding cutting fruits and vegetables. These motions are parameterized 
by some technique and repetitions to achieve the agents objective. 
FoodCutting does not require fully available knowledge
about the agents environment, instead the robot should be capable of recognizing certain objects for cutting 
operations. 

BakeBot realised on the PR2 robotics platform attempts to achieve baking cookies. An implementation of locating relevant things for MixingTasks
like a bowl and ingredients has been realised for semi-structured environments. An algorithm to perform mixing motions has been implemented as well, to mix 
ingredients with different characteristics into an uniform dough. These motions are limited to a simple circular and linear mixing motion.
The authors follow an bottom-up approach which is inherently motion driven rather than a
symbolic one which attempts to break down high level tasks into executable low level motions. 

FluidLab is a simulation environment for different kinds of manipulation tasks regarding liquids. Its underlying engine uses differentiable physics 
enabling reinforcement learning and optimization techniques in manipulation to utilize the engine, to achieve several tasks 
including liquids and solids, like mixing tasks. 

Our mixing approach will be most similar to the FoodCutting approach, in which we model symbolic knowledge about how to perform
mixing tasks, which technique should be used and infering parameters for the execution of the underlying motion.



    \chapter*{Introduction to Hardware and Software components}
    This chapter serves as an introduction to the frameworks, software, and libraries we use. The work we present can be simplified explained as a knowledge base containing parameters that a robot can use to perform specific actions.

    Firstly, we would like to introduce the robotic component, which acts as the interface between the knowledge base we have created and the robot, followed by an introduction to Ontologies and Knowledgebases.
        \section*{Robotic section: ROS and PyCram}
    
    The main framework we use is PyCram, which serves as an interface for various software components such as knowledge, perception, or manipulation. This framework utilizes another framework, ROS, to communicate with the different robot components. These two frameworks are now introduced in the following.    
    \subsection*{ROS}
    \begin{wrapfigure}{r}{0.5\textwidth}
        \centering
        \includegraphics[width=0.48\textwidth]{Graphics/ROS.jpg}
    \end{wrapfigure}
    ROS, which stands for Robot Operating System, is an open-source middleware framework designed to develop and control robots. Despite its name, ROS is not a traditional operating system but rather a set of software libraries and tools that help in building and managing robot software. It provides a standardized and modular approach to developing robotic systems, allowing for easier collaboration and code reuse in the robotics community.
    Key features and components of ROS include:
    \begin{itemize}
        \item Nodes: ROS systems are organized into nodes, which are individual processes that perform specific tasks. Nodes communicate with each other by passing messages over topics, creating a decentralized and modular architecture.
        \item Topics: Nodes exchange data through topics, which are named buses over which messages are passed. This publish/subscribe communication model allows for asynchronous and loosely coupled interactions between nodes.
        \item Launch files: ROS uses launch files to specify how to start multiple nodes and configure the system. This helps simplify the setup of complex robotic systems.
        \item Master: The ROS Master is responsible for managing the communication between nodes by keeping track of publishers, subscribers, and services. It facilitates the discovery and connection of nodes within a ROS network.
    \end{itemize}

    \subsection*{Pr2}
	Introduced in 2010 by Willow Garage (https://robotsguide.com/robots/pr2), the PR2 stands as an advanced research robot. Boasting multiple joints and 20 degrees of freedom, this robot excels in autonomous navigation and the manipulation of a diverse array of objects, making it an ideal choice for our specific needs. Additionally, it is equipped with a HeadStereoCamera that can be used to perceive the surroundings.
	
    \begin{figure}[H]
    \includegraphics[scale=0.1]{Graphics/pr2.jpg}
    \end{figure}
    \subsection*{PyCram}
    VON PYCRAM:
	PyCRAM is a toolbox for designing, implementing and deploying software on autonomous robots. The framework provides various tools and libraries for aiding in robot software development as well as geometric reasoning and fast simulation mechanisms to develop cognition-enabled control programs that achieve high levels of robot autonomy.
    PyCRAM is developed in Python with support for the ROS middleware which is used for communication with different software components as well as the robot.
    
    VON CRAM: CRAM (Cognitive Robot Abstract Machine) is a software toolbox for the design, the implementation, and the deployment of cognition-enabled autonomous robots performing everyday manipulation activities. CRAM equips autonomous robots with lightweight reasoning mechanisms that can infer control decisions rather than requiring the decisions to be pre-programmed. This way CRAM-programmed autonomous robots are much more flexible, reliable, and general than control programs that lack such cognitive capabilities. CRAM does not require the whole domain to be stated explicitly in an abstract knowledge base. Rather, it grounds symbolic expressions in the knowledge representation into the perception and actuation routines and into the essential data structures of the control programs. 

    \begin{figure}[H]
    \includegraphics[scale=1.5]{Graphics/cram-language-architecture.png}
    \end{figure}
    \section*{Knowledge Section: Ontologies and Rules}
    The parameters inferred for various robot actions come from a knowledge base. In the following, the principle of an ontology, as well as the concept of rules, which play a crucial role in parameter inference, will be introduced.
    \subsection*{Ontology}
	Ontologies are structured frameworks that provide a formal representation of knowledge within a specific domain. They play a crucial role in knowledge representation, facilitating the organization and sharing of information in a way that is both machine-readable and understandable by humans. 

    \begin{figure}[H]
        \includegraphics[scale=0.8]{Graphics/ontology.jpg}
        \end{figure}
	Key components of ontologies include:

	\begin{itemize}
		\item Concepts/Classes: These represent abstract or concrete entities within a domain. For example, in a medical ontology, "Patient" and "Disease" might be classes.
		\item Properties/Roles: These define the relationships between concepts. For instance, in a social network ontology, "FriendOf" could be a property connecting individuals.
		\item Instances/Individuals: These are specific members or examples of a class. In a geographical ontology, "New York City" and "Paris" could be instances of the class "City."
		\item Axioms: These are statements that describe the properties and relationships of the entities within the ontology. Axioms help define the logic and rules governing the domain.
		\item Hierarchy: Ontologies often organize concepts into a hierarchy, with more general concepts at the top and more specific ones below. This hierarchical structure aids in categorization and understanding.
		\item Inference Rules: These rules define how new information can be derived from existing information in the ontology. Inferences help systems reason and make deductions based on the knowledge encoded in the ontology.
	\end{itemize}
	We utilize ontologies in knowledge representation and reasoning systems to empower the robot with the ability to comprehend and handle information in a structured fashion for our specific objectives.

	\subsection*{SWRL}
	\label{sec:SWRL}
	SWRL, which stands for Semantic Web Rule Language, is a rule language that allows users to define rules about the relationships between classes and individuals in ontologies represented in the Web Ontology Language (OWL). SWRL is part of the W3C's Semantic Web technology stack and is designed to be used in conjunction with OWL to express complex relationships and infer new information based on existing knowledge.

	SWRL Rules have a specific syntax and consist of two main components:
	\begin{itemize}
		\item Antecedent (Body): This part of the rule specifies the conditions or constraints that must be satisfied for the rule to be applicable. It describes the current state of the ontology that triggers the rule.
		\item Consequent (Head): This part defines the actions or inferences that should be taken if the conditions specified in the antecedent are satisfied. It describes the changes or additional information that should be inferred when the rule is triggered.
	\end{itemize}
	SWRL supports various built-in predicates and functions, and users can create their own custom rules to suit their specific ontology. Some common elements in SWRL rules include:
	\begin{itemize}
		\item Individuals: Refers to specific instances of classes in the ontology.
		\item Class and Property Relationships: Describes relationships between classes and properties in the ontology.
		\item Built-in Predicates and Functions: Includes operations such as arithmetic, string manipulation, and comparison functions that can be used in the rule conditions.
	\end{itemize}
	Here's a simple example of a SWRL rule:
	\newline
	$Person(?x) ^ hasChild(?x, ?y) -> Grandparent(?x, ?z) ^ hasChild(?y, ?z)$
	\newline
	In this example:

	If an individual (?x) is a Person and has a child (?y),
	
	Then, infer that the individual (?x) is a Grandparent of another individual (?z), and the child (?y) is the parent of (?z).
	
	SWRL rules are useful for expressing complex relationships and constraints within ontologies, enabling automated reasoning systems to make inferences and derive new knowledge from existing data.
	\section{Libraries}
	\label{sec:Libraries}
	\subsection*{OWLReady}
	\label{sec:OWLReady}
	One important library used for our Implementation is OWLReady. "OWLReady" is a Python library designed for ontology-oriented programming. It facilitates the development, manipulation, and querying of ontologies using the Web Ontology Language (OWL), a standard for representing knowledge in a machine-readable format. OWLReady simplifies ontology-related tasks by providing a convenient and object-oriented interface for working with OWL ontologies in Python.

	Key features of the OWLReady library include:

	\begin{itemize}
		\item Object-Oriented Programming (OOP): OWLReady adopts an object-oriented approach, allowing users to interact with ontology entities as Python objects. This makes it more intuitive for developers familiar with Python's OOP principles.
		\item Ontology Loading and Parsing: The library supports the loading and parsing of OWL ontologies, making it easy to access and manipulate ontology data within Python scripts or applications.
		\item Class and Individual Manipulation: OWLReady provides functionality for creating, modifying, and querying classes and individuals within an ontology. This allows for dynamic and programmatic management of ontology content.
		\item Reasoning Support: Depending on the version and features, OWLReady may offer support for reasoning tasks. Reasoning involves deducing implicit information based on the logical relationships defined in the ontology.
		\item Integration with RDFLib: OWLReady may integrate with RDFLib, another Python library commonly used for working with Resource Description Framework (RDF) data. This integration enhances the capabilities of handling semantic data.
	\end{itemize}
	\newpage
	\subsection*{WikiHow}
	
	\begin{wrapfigure}{r}{0.5\textwidth}
        \centering
        \includegraphics[width=0.48\textwidth]{Graphics/WikiHow Article Structure.png}
    \end{wrapfigure}
	WikiHow is an extraction tool, that can collect informations from the WikiHow Corpus. The goal of this tool is to analyse a WikiHow corpus using basic NLP techniques to gather information about Everyday tasks like "Pouring", "Cutting" or "Discarding". These information should support cognitive robots in understanding and parameterizing these tasks to better handle unknown tasks, working in underspecified environments and handling common task-object combinations.
	Jedes Verb hat eine eigene Klasse, worin das Verb und die zusätzlich gewünschten Hyponyme/Synonyme definiert sind. Diese Verben dienen das als Schlüsselwort, wonach in den WikiHow Artikeln letzendlich gesucht wird. Außerdem kann man verschiedene Parameter einstellen, wie zum Beispiel das Ausschließen verschiedener Kategorien, welche für die Suche nicht von Relevanz sind.

\chapter*{Data acquisition}

A part of this master's thesis involves creating a knowledge base that includes certain queryable parameters, allowing the robot to perform specific actions. The initial step in building the knowledge base is to gather data. In this chapter, we elaborate on our data acquisition strategy.
\section*{Task variations}
	As our main focus is to represent the knowledge about mixing, first we had to acquire the different types and variations of mixing in order to create a complete Knowldegerepresentation. The first step in acquiring the needed data, was to acknowledge which task varations of mixing are actually important. 
  So we had to analyze the word \textit{Mixing} and its hyponyms. But first we have to ask ourselves: \textbf{What is Mixing?}
  \subsection*{What is Mixing?}
  The definition provided by the Oxford Dictionary \cite{Oxford} is as follows:
  
  \textbf{Definition:} To put together or combine (two or more substances or things) so that the constituents or particles of each are interspersed or diffused more or less evenly among those of the rest; to unite (one or more substances or things) in this manner with another or others; to make a mixture of, to mingle, blend.

  Ultimately, this definition conveys that mixing requires at least two elements or substances, which are then combined (evenly) with each other, resulting in a (new) substance.
  This definition is general and can be applied to various contexts. For our work, the aspect of cooking or mixing different (cooking) ingredients is important. Therefore, we consider some hyponyms of the verb \textit{mixing} irrelevant for our cause and do not take them into account.
  An adapted definition for our work could be: 
  
  \textbf{Definition:} Mixing is the combination of various (cooking) ingredients through different motions in a container.
  
  \subsection*{Hyponyms} 
	Hyponyms are subordered words of a given word, for example one hyponym of mixing could be beating. 
  To conduct this analysis, one can utilize tools from various websites, such as FrameNet\cite{FrameNet} and WordNet\cite{WordNet}. These platforms provide users with the ability to search for specific words and obtain various associations for those words, including synonyms, acronyms, or, crucial for our case, hyponyms.	
  A full list of the considered hyponyms can be seen in table \ref{tab:hyponyms}.
  \subsubsection*{WikiHow extraction}
  For all those hyponyms, we delegated a WikiHow extraction search (see: \nameref{sec:WikiHow}) which should show us, how many times one of these words occur, in the context of cooking.
	To execute the search, we define a new Class \textit{MixingVerb}, which represents the verb \textit{mixing} and its hyponyms.
  \begin{figure}[H]
    \includegraphics[scale=0.4]{Graphics/MixingVerbClass.png}
    \caption{\textit{MixingVerb} definition}
    \label{fig:mixingverb WikiHow}
  \end{figure}
    Since we have now defined the desired verbs that we ultimately want to search for, let's adjust some search parameters. The crucial parameters involve filtering the categories; in our case, we want to focus on mixing in the cooking domain. Therefore, we filter out all articles that do not exist in the Food and Entertaining category.
    \subsubsection*{Hyponyms occurance}
    For each defined verb, a search is initiated to determine how often this verb appears in the WikiHow articles \cite{wikihow}. This is done to ascertain which verbs are ultimately relevant for our implementation and which ones we can exclude, as they are infrequently used in everyday language.
    In the table below the results can be seen.
    \begin{table}[H]
        \centering
        \begin{tabular}{|c|c|}
          \hline
          \textbf{Hyponym} & \textbf{Occurance}  \\
          \hline
          Mix & 5300 \\
          \hline
          Amalgamate & 0 \\
          \hline
          Beat & 956 \\
          \hline
          Blend & 1041 \\
          \hline
          Coalesce & 1 \\
          \hline
          Combining & 3591  \\
          \hline
          Coommingle & 0 \\
          \hline
          Compound & 0 \\
          \hline
          Conflate & 0 \\
          \hline
          Folding & 821 \\
          \hline
          Fuse & 17 \\
          \hline
          Intermix & 0 \\
          \hline
          Join & 53 \\
          \hline
          Jumble & 0 \\
          \hline
          Lump & 7 \\
          \hline
          Merge & 6 \\
          \hline
          Pair & 352 \\
          \hline
          Stir & 6027 \\
          \hline
          Unify & 2 \\
          \hline
          Unite & 2 \\
          \hline
          Whip & 863 \\
          \hline
          Whisk & 2267 \\
          \hline
          
    
        \end{tabular}
        \caption{Mix synoyms/hyponyms occurance}
        \label{tab:hyponyms}
      \end{table}
      
  \subsubsection*{Further examination and conclusion}
  This search is ultimately intended to provide us with information on which tasks we want to represent in the knowledge base. Therefore, we decide on certain tasks based on two conditions: frequency and executability. The first condition is easy to understand; tasks that do not occur or occur very rarely are not considered. The second condition relates to the executability of the task in the context of robot movements. Additionally, some tasks with relatively high frequency are also examined more closely, as in English, the past tense is used as an adjective under certain circumstances.

Taking into account the first condition, the following tasks are not considered: \textit{Amalgamate, Coalesce, Comingle, Compound, Conflate, Fuse, Intermix, Join, Jumble, Lump, Merge, Unify, and Unite}.

The second condition excludes another task: \textit{Blend. Blend} is mostly used in the context of a blending machine, which is not handled by the robot. Without this machine, the blending execution cannot be performed correctly, so this task is excluded for us.

Upon closer examination, we will also not consider the verb \textit{Whip} because it is mostly used as an adjective for ingredients, such as whipped cream. This highlights that only the verb \textit{Whip} has relatively low frequency. The same applies to \textit{Pair}, where the past tense is used to describe a combination of different ingredients, such as wine paired with cheese.

Thus, the tasks we consider are: \textit{Mix, Combine, Beat, Fold, Stir, and Whisk.}

\section*{Task analysis and defintion}
Now that we have selected the tasks, they need to be analyzed to understand the context in which they are ultimately used. 
Our goal is for the robot to perform these tasks in a manner similar to how a human would. 
To achieve this, in the next step, we need to closely examine these tasks. 
It is recommended to analyze videos on WikiHow or other sources where these tasks are presented as activities. 
The analysis involves observing the movements associated with each task. 
We present these analyses in tabular form below. The examination includes the task itself, the respective ingredients being processed, the tools used for it, and the container in which the task is carried out.
The tasks will be ennumerated, and we will provide the full table with the actual video source in the appendix (HERE REFERENCE TO APPENDIX).
    \begin{table}[H]
    \centering
    \begin{tabular}{|c|c|c|p{4,5cm}|p{4,5cm}|}
        \hline
        \textbf{Task} & \textbf{Tool} & \textbf{Container} & \textbf{Ingredients} & \textbf{Description} \\
        \hline
        Beating$^1$ & Whisk & Bowl & Egg yolk (Wet ingredient) & circular, swirling wildly around the bowl \\
        \hline
        Stirring & Whisk & Bowl & Beaten Egg Yolk (Wet), Parmesan(Powder) and Pepper (Powder) & Circular, from the inside to the outside. \\
        \hline
        Whisk & Fork & Bowl & Eggs (Wet) & Circular but also straight, wildly motion. \\
        \hline
        Mixing & Spatula & Pan & Eggs, melted butter (Wet) & Circular, from the inside to the outside, also diving. \\
        \hline
        Folding & Spatula & Pan & cooked eggs in melted butter (Wet) & Gently motion from the outisde to the inside straight, then moving about 90 degree before going to the inside again. \\
        \hline
      \end{tabular}
    \caption{Video analysis}
    \label{tab:videoanalysis}
  \end{table}


After a thorough analysis of the videos and the information provided in them, we conclude that the executed movement is not only related to the task but also influenced by the ingredients used. However, some tasks are deterministic in the sense that the movement is performed regardless of the specific ingredients.

In the following, we aim to structure the extracted information from the videos and present our findings.

\subsection*{Video analysis conclusion and results}

Based on the extracted information, we conclude that for our goals, the following aspects, in addition to the task, are important and will be further considered:
\begin{itemize}
  \item Ingredients: The ingredients play a crucial role in the movement decision associated with the tasks and will be defined more precisely.
  \item Tools: The tools used may not be decisive in the movement decision.
  \item Container: As conducted with the tools, the container wont play a crucial in the motion decision.
  \item Motions: The motions ultimately represent the movement of the robot for our implementation. These movements are extracted from the videos and defined in alignment with robot motions.
\end{itemize}

\subsubsection*{Ingredients}
Through the video analysis , we come to the conclusion that the nature of ingredients is important. 
Primarily, ingredients can be divided into two main categories: \textit{Dry} and \textit{Wet} \cite{Ohene2017}. This division becomes crucial, especially regarding the measurement of quantities for each type of ingredient.

For our case, a somewhat finer categorization of ingredients is necessary to map the various movements sensibly to the given types. In addition to \textit{Wet} ingredients, we introduce a new category: \textit{Liquid} ingredients. 
This corresponds to ingredients falling under the \textit{Wet} ingredients but with a liquid state, such as milk, water, and oil. This is significant because some movements differ when the given ingredients are \textit{Wet} or \textit{Liquid}.

Another category we include is the \textit{Solid} ingredients category. This differs from the \textit{Dry} category in that it consists of solid components, while we define the \textit{Dry} category to include powdery substances. The \textit{Solid} category encompasses foods like vegetables, fruits, and meats. Through these distinctions, we can create a definition refining the ingredient categories.

\textbf{Definition:}
Ingredients are primarily divided into \textit{Dry} and \textit{Wet}, with a need for finer differentiation. An abstracted category called \textit{Liquid} is introduced to encompass liquid states such as milk and water. An additional category, \textit{Solid}, distinguishes itself from \textit{Dry} by including solid components like vegetables and meat. These distinctions allow for a precise definition of ingredients, enhancing the analysis of mixing motions in the context of cooking/baking and their representation.
In the initial version, the ingredients we defined are for example:
\begin{itemize}
  \item \textit{Liquid: Milk, Oil, Water, Vinegar, Vanilla Extract, Sauces}.
  \item \textit{Wet: Egg White, Egg yolk, Butter, Whipped Cream}.
  \item \textit{Dry: Flour, Salt, Sugar, Baking Soda, Cocoa Powder}.
  \item \textit{Solid: Onions, Pork, Chicken, Minced meat, Bacon}.
\end{itemize}
These ingredients are commonly used in the analyzed videos and are also very common ingredients for baking and cooking recipes.


\subsubsection*{Containers and Tools}
\label{sec:ContainersAndToolsAcquisition}
Throughout the analysis of wikihow videos common tools and containers were identied during mixing task execution.
The most dominant occuring container was a bowl, followed by a pan, cup and then lastly a mug. Some common tools were used, like a whisk, spoon, spatula and mixer.
Since all of the containers can be used to serve or store food, this can be summarised in a single concept - crockery. \textbf{Crockery} is a concept in the \textit{SOMA-Home}
ontology, by subsuming all of the subclasses of \textbf{Crockery}, we acquire in an instant many containers for mixing. Identified tools are partially represented by \textit{SOMA-Home},
mainly cutleries like spoons are belonging to this concept. Using \textbf{Cutlery} we can acquire a set of tools for mixing. Tools like whisks and mixers neither belong 
to the \textbf{Cutlery} concept nor have a representation in \textit{SOMA}, thus a higher level concept can't be acquired for these tools, to retrieve a set of tools resembling
kitchen tools. 

\paragraph*{Why is SOMA used}
Going through all available wikihow videos is very tedious and time consuming. By identifying common occuring containers and tools, 
a generalization can be made and this is where \textit{SOMA} helps. \textit{SOMA-Home} is a taxonomy modelling concepts for a kitchen environment.
Top level concepts from \textit{SOMA-Home} can help us identify specific tools and containers without going through dozens of videos. 
Another advantage is, when \textit{SOMA-Home} is extended with new concepts, those concepts can be easily included into the mixing task procedure.   


\subsubsection*{Motions}
	From this videos we extracted informations about the executed motions. The following motions can be defined:
	\begin{itemize}
		\item \textbf{Circular}: Moving the tool in a defined circular movement in the container, not changing the radius during execution
		\item \textbf{Whirlstorm}: Moving from the inside to the outside of the container with the tool, by circulating in an incremented radius.
		\item \textbf{Folding}: Gently motion, where you start from the outside, moving one straight line to the inner side, then picking the tool up and going to the initial state before moving the tool for about 90 degrees, then going back in a straight line to the inner side of the container again.
		\item \textbf{VerticalCircular}: Imagine a line which can be seen as the diameter of the container, from this line one can define certain regions on which you move the tool circular from side to side. This motion is used by the beating task.
		\item \textbf{CircularDivingToInner}: Starting from the outerside, moving the tool in the container arround its edge for about 270 degrees, before diving to the middle of the container. This motion is used by certain tasks where is required to turn the ingredients over.
	\end{itemize}

\section*{Data Acquision Conclusion}

In the data acquisition phase, our focus was on creating a foundation of the necessary data and information required for a robot to perform various mixing tasks. After identifying the tasks, we were able to analyze videos to examine the movements associated with each task. We concluded that not only the task itself but also the selection of ingredients is crucial for determining the movement.

The ingredients are divided into four different categories, and we were able to define five motions that the robot should perform in a manner consistent with everyday use. In the next chapter, we will present how we intend to represent this knowledge and create a system that enables the robot to know which movement to execute based on a given task and set of ingredients.
\newpage

\chapter{Data Representation}
\label{chap:Data_representation}
\begin{figure}[H]
    \includegraphics[scale=0.25]{Graphics/structure_overview2.jpg}
\end{figure} 
In this chapter we will present you the mixing knowledgegraph using \textit{OWL} as basis to formally define knowledge.
This knowledge base defines domain knowledge about mixing, which encompasses several aspects derived from the data acquisition.
Using assertional knowledge such as the task to execute, the set of ingredients as input for a mixing task, 
and additional knowledge like which container to use, the knowledge base can infer facts about the motions that should be executed.
Our goal is not to teach a robot how to execute a specific mixing motion, but rather infer a motion and relevant parameters for this motion. 
\section{The Knowledgebase}
Our knowledgebase consists of the superclasses \textit{Ingrediet, DesginedTool, Task} and \textit{Motion}. These are the concepets which we dervied from the \nameref{chap:Data_acquisition}. The following image illustrates the \textit{Mixing} ontology hierarchy.
\begin{figure}[H]
    \includegraphics[scale=0.5]{Graphics/classHierarchy/taxonomy.png}
    \centering
    \caption{Mixing Ontology Taxonomy}
\end{figure}

\subsection{Ingredients}

\begin{figure}[H]
    \includegraphics[scale=0.45]{Graphics/classHierarchy/ingredients_hierarchy.png}
    \centering
    \caption{Ingredients Hierarchy in Mixing Ontology}
\end{figure}

Ingredients are primarily divided into \textit{Dry} and \textit{Wet}, with a need for finer differentiation.
Wet ingredients can be further differentiated into \textit{Liquid} and \textit{Semi-Liquid}, where liquids include examples like water and semi-liquids include examples like egg whites or yolks. 
Dry ingredients, on the other hand, are divided into the subcategories \textit{Solid} and \textit{Powder}. 
Solid ingredients include whole vegetables and meat, while powder ingredients are not solid but rather granulated. 
This hierarchy of ingredients is modeled with a view towards the cooking domain and enables us to formulate rules to infer motions to perform. 
There does not exist a single hierarchy of ingredients, but rather many hierarchies modeling different aspects of ingredients.

\begin{itemize}
    \item \textit{Liquid: Milk, Oil, Water, Vinegar, Vanilla Extract, Sauces}.
    \item \textit{Semi-Liquid: Egg White, Egg yolk, Butter, Whipped Cream}.
    \item \textit{Powder: Flour, Salt, Sugar, Baking Soda, Cocoa Powder}.
    \item \textit{Solid: Onions, Pork, Chicken, Minced meat, Bacon}.
  \end{itemize}

For each specific ingredient, a respective \textit{FoodOn} class is used. By using \textit{FoodOn}, we directly include all available subclasses of the 
\textit{FoodOn} class once the ontology is imported into the \textit{mixing} ontology.
Different variations of the same ingredient are then covered in case a finer differentiation is needed. 
Extending the hierarchy of ingredients can be done by for example integrating \textit{FoodOn} classes.  

\subsection{Tools and containers}
\label{sec:toolsandcontainers}

\begin{figure}[H]
    \includegraphics[scale=0.45]{Graphics/classHierarchy/tools_hierarchy.png}
    \centering
    \caption{Tools and Container Hierarchy in Mixing Ontology}
\end{figure}

Using the identified tools and containers from the data acquisition, we partially use \textit{SOMA} to define parts of the containers and tools hierarchy.
We use the class \textit{DesignedTool} from \textit{SOMA-Home}, which consists of all kinds of container and tool classes. 
\textit{Crockery}, which is subsumed by \textit{DesignedTool}, is included with each container we identified. 
To include all kinds of cutlery, we use the class \textit{Cutlery}, which subsumes the classes \textit{Fork} and \textit{Spoon}. 
Mixers and whisks are not represented in \textit{SOMA-Home}, so we introduce a new class, \textit{KitchenTool}, which includes both of these tools.

Using \textit{SOMA-Home} has a few advantages:
\begin{enumerate}
    \item Using an existing hierarchy can be reused in other taxonomies.
    \item If \textit{SOMA-Home} is extended with new containers and tools, these concepts can be easily included.
    \item The \textit{mixing} ontology can be used by \textit{Soma-Home}, because both ontologies share classes.
\end{enumerate}

\subsection{Tasks}

\begin{figure}[H]
    \includegraphics[scale=0.45]{Graphics/classHierarchy/task_hierarchy.png}
    \centering
    \caption{Task Hierarchy in Mixing Ontology}
\end{figure}

Mixing task variations are subclasses \textit{Task} superclass, where \textit{Task} is \textit{DUL} concept. The tasks consists of \textit{Mixing}, which can be regarded as the umbrela term of all tasks, \textit{Stirring} which is mostly a task that includes a circular motion, \textit{Beating} will be mostly used in context with eggs and other wet ingredients, \textit{Folding} which represents a gentle type of mixing and the last task is \textit{Whisking} which is similar to the beating task.
While some of these tasks have only one motion that can be dervied, like the \textit{Folding} task, we discovered that the others tasks can have multiple motions associated to them. This decision is influenced by the set of ingredients on which the motion will be performed

\subsection{Motions}
VON NASER
\begin{figure}[H]
    \includegraphics[scale=0.45]{Graphics/classHierarchy/motions_hierarchy.png}
    \centering
    \caption{Motion Hierarchy in Mixing Ontology}
\end{figure}


The top level class \textit{Motion} is the class of all possible motions which can be performed. MixingMotion a subclass of motion, are all mixing motions 
which were found during data acquisiton. To make a distinction between one singular mixing motion and motions consisting of at least two motions
we created the class \textit{CompoundMotions}, which defines a sequence of motions to be performed.
\textit{MixingMotion} is comprised of the following motions:

\begin{itemize}
    \item \textit{Circular Motion}: Simple Circular Motion
    \item \textit{Circular Diving To Inner}: Circular Motions including diving moving to the center of a container
    \item \textit{Folding Motion}: Motion executed as line with back and forth on multiple areas inside the container.
    \item \textit{Horizontal Elliptical Motion}: Multiple ellipse motions around the container, which resembles mixing `wildly`
    \item \textit{Whirlstorm Motion}: Whirlstorm or spiral like motion
\end{itemize}

Each motion has class restrictions with different kinds of motion parameters.

The following motion parameters are modelled in the \textit{mixing} ontology:

\begin{itemize}
    \item \textit{Radius lower Bound Relative}: The smallest possible radius relative to an object.
    \item \textit{Radius Upper Bound Relative}: The maximum possible radius relative to an object.
    \item \textit{Folding Rotation Shift}: An angle value of how much the folding line has to be rotated around .
    \item textit{Repetetive Folding Rotation Shift}: After the folding motion has been executed this parameter rotates all folding lines.
     to cover similar areas of a container. 
    \item \textit{Ellipse Shift}: How much the ellipse has to be moved in meters. For example, if the ellipse shift is 0.04, then the ellipse
     is shifted 4 cm in the positive or negative direction along the x or y axes.
\end{itemize}

\textit{CompoundMotion} is comprised of the following classes:
\begin{itemize}
    \item \textit{WhirlstormThenCircularDivingToInner}: Execute whirlstorm motion followed by circular diving to inner motion.
    \item \textit{WhirlstormThenHorizontal}: Execute whirlstorm motion followed by horizontal elliptical motion.
\end{itemize}

Each of these compound motions have an OWL expression stating which motion is executed and which motions if following:

\[
	\begin{aligned}
	&\text{Motion}\\
    &\text{ and (hasMotion some}\\
    &\text{      (WhirlstormMotion}\\
    &\text{       and (followedBy some HorizontalEllipticalMotion)}\\
	&\text{     and (followedBy exactly 1 Motion)))}\\
	\end{aligned}
\]

This compound motion is equivalent to a \textit{Motion}, which has a motion \textit{WhirlstormMotion} followed by 
\textit{HorizontalEllipticalMotion}, where \textit{WhirlstormMotion} is exactly followed by one motion. 

How each of these motions is related to a task and the set of ingredients will be explained in the next section.

\section{Rules - Grafiken anpassen}
In order to infer the right motion based on the ingredients and task input, rules have to be defined. This rules are \nameref{sec:SWRL}-rules.
In this section we want to illustrate the inference on a high level.
The rules can be thoguht as if conditions, which will then result in a motion, 
for example, if we regard the combination of the task \textit{Mixing} and the ingredient type \textit{Liquid}, we infer the motion \textit{Whirlstormmotion}. 
This can be done for every task and ingredient combination. 
The inference can be illustrated with decision trees which will be shown in the following section for every task and ingredient type combination available. 

\subsection{Mixing}
\begin{figure}[H]
\includegraphics[scale=0.18]{Graphics/MixingDecisionTree.jpg}
\caption{Mixing decision tree}
\end{figure}
\textbf{Definition:} In the context of baking or cooking, a mixing task refers to the process of combining multiple ingredients thoroughly to create a homogeneous mixture. The goal is to distribute the ingredients evenly, ensuring that each component contributes to the overall texture, flavor, and consistency of the final dish or baked good. Mixing is a fundamental step in many recipes and is essential for achieving a balanced and cohesive result (Quelle).


As can be seen from the illustration, the mixing task can primarily be mapped to the Whirlstorm motion. This is not surprising when considering the definition, as this motion results in the uniform distribution of all ingredients in the container.
\subsection{Stirring}
\begin{figure}[H]
    \includegraphics[scale=0.18]{Graphics/StirringDeisionTree.jpg}
    \end{figure}
\textbf{Definition:}
In the context of baking or cooking, a stirring task involves using a utensil, such as a spoon, spatula, or whisk, to agitate and circulate the ingredients within a mixture. The purpose of stirring is to achieve a uniform distribution of ingredients

In comparison to \textit{Mixing}, the \textit{Stirring} task, depending on the ingredients, maps to a broader range of motions. In addition to the \textit{WhirlstormMotion}, the \textit{Circular Motion} is used here for the first time. This is particularly important when considering the \textit{Stirring} task with the ingredient type \textit{Liquid}, as one does not want the ingredients to be whirled, as would be the case with the \textit{Whirlstorm Motion}, but rather just stirred.
\subsection{Beating}
\begin{figure}[H]
    \includegraphics[scale=0.18]{Graphics/BeatingDecisionTree.jpg}
    \end{figure}
\textbf{Definition:}
In the context of baking and cooking, a "beating" task refers to the process of vigorously stirring or mixing ingredients to achieve a specific texture or consistency. Beating is often done to incorporate air into the mixture, create smooth and uniform blends, or alter the physical properties of certain ingredients.

In addition to the Whirlstorm Motion, the Horizontal Eliptic Motion also predominates here. This can be derived from the definition as well, as this motion requires a wild mixing style, to which the Horizontal Eliptic Motion aligns.
\subsection{Whisking}
\begin{figure}[H]
    \includegraphics[scale=0.18]{Graphics/WhiskingDecisionTree.jpg}
    \end{figure}
\textbf{Definion:}In the context of baking and cooking, a "whisking" task involves using a kitchen utensil called a whisk to mix, blend, or beat ingredients. A whisk typically consists of wire loops or a coil attached to a handle, and it is designed to incorporate air into mixtures, break up clumps, and create a smooth and uniform texture.

\subsection{Folding}
\begin{figure}[H]
    \includegraphics[scale=0.18]{Graphics/FoldingDecisionTree.jpg}
    \end{figure}
\textbf{Definition:}
In the context of baking and cooking, a "folding" task refers to a gentle mixing technique used to incorporate ingredients without deflating or destroying the air bubbles that have been created. Folding is often employed when combining a lighter mixture (such as whipped cream or beaten egg whites) with a denser one (such as a batter or a heavier mixture). The goal is to maintain the desired texture, lightness, or fluffiness in the final dish.

Theoretically, we wouldn't need to graphically represent the Folding Task since each task and ingredient combination maps to a single motion, the Folding Motion. The graphic was nevertheless included for completeness. The Folding Task requires a specific movement that does not remove the air in the mixture, and any other movement would fail.


\chapter{Motions}
\label{chap:Motions}
\begin{figure}[H]
    \includegraphics[scale=0.3]{Graphics/overview_3.jpg}
\end{figure}
In this chapter we'll describe on an abstract level how the motions are implemented in pycram2.
To simplify the implementation of motions, we hold the following assumptions about the motions which are performed on containers.
Firstly we assume, the container is symmetric in the x,y direction. We don't consider the descent of the container, we keep the height for the tool
constant over the performed motion. We don't have collision detection.

\section{Circular Motion}
The circular motion is generated on a circular plane. 
By having a center coordinate = $(x,y)$, which is the center coordinate of 
the container to execute the Circular Motion, we generate a set of coordinates
with the following functions: 

$circleX = containerCenterX + radius * cos(radian)$

$circleY = containerCenterY + radius * sin(radian)$

We retrieve the respective radians using the following function:

$ angle2radian = angle * \frac{\pi}{180} $

Here, we pick a sequence of angles starting from
0 to 360 degrees convert these into radians in order to generate points lying on a circle.

INSERT Plot

\section{Folding Motion}
The folding motion is not a circular motion but rather a linear motion. In its current implementation, the motion is a singular line having where one endpoint is on the edge of the container
the other is center of the container. While the motion in itself is a straight line going through the container, the motion is executed by traversing from one end of the container to the center,
and then repeating the motion flipped - from center to one of the end, which creates a folding effect.
This line partially covers the whole container, therefore the direction has to be realigned in order to mix different sections of the container.

By constructing a 2D rotation matrix for some angle, where the angle determines the shift of the line $\theta$:
\[R(\theta) = \begin{bmatrix}
    \cos(\theta) & -\sin(\theta) \\
    \sin(\theta) & \cos(\theta)
     \end{bmatrix}
 \] 

a transformation onto the line can be applied, which shifts the line using this function: 

$\begin{bmatrix} x' \\ y' \end{bmatrix} = R(\theta) * \begin{bmatrix} x \\ y \end{bmatrix}$

The folding motion is executed multiple times, in that the motion is covering multiple parts of the container, depending on what 
the value of the angle is to shift the folding line. 

Once the folding motion has finished, by shifting the folding line another time with an angle $\phi$
we realign the line, so that different parts of the container can be folded by using angle $\theta$ again.

INSERT Plot

\section{Horizontal Elliptic Motion}
The horizontal elliptic motion is executed on a horizontal plane and involves multiple ellipses.
Using the function to compute x,y coordinates on a circle, we use one constant radius and one radius which is changing. 

By using the formular to compute points on a circle, we adjust the radius for 
x,y, where one of the values is set to 0.01, the other one is sampled evenly from the interval $semi\_major\_radius = (radius\_upper\_bound,radius\_upper\_bound / 2 ).$
We take the maximum sampled radius which fulfills the following condition. 
$\sqrt{(x\_{i} - center\_x)^2 + (y\_{i} - center\_y)^2} < radius_upper_bound$
Additionally we decrement and increment y coordinates. If the condition: $\sqrt{(y\_{i} - center\_y)^2} > radius_upper_bound$
is met we change from decrementing to incrementing and vice versa until this condition is fulfilled again.

\section{Whirlstorm Motion}
The whirlstorm motion uses the function to generate points lying inside a circle, with differing radiuses, over the course of the motion. 
Using an upper bound radius and a lower bound radius, we create an evenly spaced interval between those radiuses, to use them for coordinate generation. 
Starting from the upper bound to the lower bound translates to performing the motion from the edge to the center or close to the center, depending on the 
values of upper and lower bound radius. 

To create a consistent and fluid motion, we introduce mirroring of the coordinates and alternating from outer to inner circular motion and vice versa.
These are visualized in these plots:




\chapter{Simulation}
In unserer Arbeit haben wir eine Wissensbasis erstellt, welche Parameter inferrieren kann, womit ein Agent verschiedene Aktionen ausführen kann. Da es sich als herausfordernd darstellt die Aktionen auf einem realen Roboter zu testen tun wir dies in einer Simulation.
Dieses Kapitel soll einen Überblick über die Simulationsumgebung geben, in der wir unsere Arbeit testen und als Proof of Concept darstellen möchten.
Zunächst wollen wir die Umgebung, samt Objekten erläutern, darauf folgend möchten wir einen Überblick über die Motion Primitives geben.
Zum Schluss werden wir einige unserer Motions showcasen und über die SimToRealWorldGab sprechen, bevor wir einen Fazit ziehen, wo wir über vorhandene Schwierigkeiten diskutieren.

\section{Simulation Environment}
Wir benutzen die Simulationsumgebung BulletWorld \nameref{sec:pybullet}, welche sowieso in dem Framework PyCRAM \nameref{sec:pycram} genutzt wird. 
In dieser Umgebung wird die Physik dementsprechend auch simuliert, welches für die Ausführung der von uns definierten/angepassten Motions vorteilhaft ist.
Da wir außerdem das Framework \textit{PyCRAM} nutzen, ermöglicht sich uns die Möglichkeit der Benutzung verschiedener Control Primitiven, welche schon vorhanden sind, wie zum Beispiel Pick and Place.
Für die Kommunikation zwischen der Simulation und Framework wird ROS1 \nameref{sec:ROS} genutzt, welches über verschiedene Nodes kommuniziert, die jeweils zum Beispiel einen Joint repräsentieren.
Über diese Nodes kann man dementsprechend Informationen zu den Joints erhalten oder selbst Befehle übertragen, welche die Parameter der Joints dementsprechend verändern würden.

Der Agent, der in der Simulation die verschiedenen Motions ausführen wird, ist der PR2 \nameref{sec:pr2}. Dieses Robotermodell verfügt über 2 bewegbare Arme, dieser Aspekt ist wichtig für unsere Arbeit, da bei der Ausführung der Motions eine Hand die Motion ausführt, während der andere Arm den Container festhält, in dem die Aktion ausgeführt wird.
Neben der Tatsache, dass das Modell des PR2 schon im Framework PyCram vorhanden ist, verfügt das Modell des PR2's außerdem auch über genügend Joints und Degrees of Freedom, um die von uns angepassten Motions auszuführen.

Der Agent agiert in einer Küchenähnlichen Umgebung, dessen Modell auch im PyCRAM Core schon enthalten ist und dementsprechend sich auch für unseren UseCase besonders gut eignet. Die Küche verfügt auch über zum Beispiel Schubladen, welche geöffnet werden können, was zu einer Annaeherung an einer realen Küche hindeutet.
Das Modell der Küche verfügt außerdem über einen Tisch, worauf schlussendlich die Aktionen des Agenten ausgeführt werden.
Weitere Möbel sind auch in dieser Umgebung enthalten, diese sind allerdings für unseren Use Case nicht von Bedeutung.

\begin{figure}[H]
    \includegraphics[scale=0.35]{Graphics/bulletworldexample.png}
    \label{fig:bulletworldexample}
    \caption{Simulation Environment containing the \textit{PR2} and \textit{Kitchen} models. }
\end{figure}

Für die Simulation nutzen wir die Objekte, welche auch in der Ontologie definiert wurden (siehe \nameref{sec:ContainersAndToolsAcquisition}).
Folgende Objekte sind vorhanden:
\begin{itemize}
	\item Container: kleine und große Bowl (in der Ontologie \textit{SaladBowl} und \textit{PastaBowl} definiert), Pfanne, Topf, Becher und Tasse.
	\item Tools: Whisk (auch wenn diese nicht in der Ontologie definiert wurde, nutzen wir sie trotzdem, da es doch ein sehr häufig vorkommnedes MixInstrument ist.), Löffel, Holzlöffel und Gabel. 
\end{itemize}

\begin{figure}[H]
    \includegraphics[scale=0.35]{Graphics/toolscontainersmodels.png}
    \label{fig:toolscontainersmodels}
    \caption{Used models for the simulation: \textit{Fork, WoodenSpoon, Spoon, Whisk, Mug, Cup, Pan, Pot, Small Bowl, Big Bowl}}
\end{figure}

PyCram bietet schon eine betrachtliche Anzahl an implementierten Aktionen, womit der Roboter manevriert werden kann. 
Unter diesen Aktionen befinden sich notwendige Navigationaktionen, sowie manipulative Aktionen wie Greifen und Platzieren. 
Außerdem bietet PyCram schon Schnittstellen bereit womit die Joints des Roboters bewegt werden können, diese Funktionen heißen dann zum Beispiel moveTorso().
BILDER CODE FUNKtIONEN

\section*{HIER STELLEN WIR UNSERE MOTIONS VOR}
Hier Vanessa fragen, wie wir ihre Motion referenzieren sollen, einfahc sagen es war vorhanden oder es detaillierten angeben?

\section*{Simulation to RealWorld gap}
\begin{itemize}
	\item Big Problem: Uncertaininty
	\item Perception Solution as first approach: RoboKudo
	\item Data acquisition: Blenderproc
	\item Model Training: YoloV8
	\item Results showing the real world Perception.
\end{itemize}

\section{Implementation in PyCram}

\subsection{Workflow}




\subsection{Asserted Knowledge}
Any robot capable of executing mixing motions need to have knowledge before executing a mixing task. 
The robot needs to know where the container and tools are located, for picking up and for navigation to Theoretically
respective places. It needs to know which ingredients have to be mixed, in simulation only their existence have to be known. 
It needs to know which task to execute, since we differentiate between different mixing tasks as discussed in this section. 


\subsection{Inferring parameters}
The robot doesn't know what kind of motion with fitting parameters it has to execute. Once all the assertional knowledge is available
to the robot, it is then capable of inferring motion and parameters through reasoning. 


\subsection{MixingActionSWRL}
This class has multiple functionalities to infer the relevant motion or compound motions to execute, with its associated 
parameters. This class needs asserted knowledge by the robot, which tool has to be used, which ingredients have to be mixed 
which task has to be executed. To enable inference on the Mixing knowledgegraph, we have to first initialize instances of 
all ingredients, task, tool and container. Additionally a motion instance is initialized using the top-level concept \textit{Motion}. 
SWRL rules are only applied onto instances, not on classes. 

Once the initialization has been completed, we run the inference, which either uses a reasoner like pellet or HermIt. 
Running the inference will cause a reclasification on the motion instance, i.e a motion becomes a circular motion or something else. 
Since we know which class it has been reclassified to, we can then extract all parameters from the respective class and set the motion to be executed as well.

The inferred knowledge is passed onto the MixingAction which executes the actual mixing motion. 








\chapter{Evaluation}
\begin{figure}[H]
  \includegraphics[scale=0.3]{Graphics/overview_5.jpg}
\end{figure}
In this chapter, we aim to present our implemented/adapted motions as a proof of concept. We consider all possible combinations of tasks and ingredients applied to every possible combination of tools and containers. Each task will showcase only one example in this chapter, with the rest available in the appendix. The respective task trees are located in Chapter \nameref{chap:Data_representation}, which illustrates which motions result from combinations of tasks and ingredients.

Through this evaluation, we aim to demonstrate that the knowledge base we have developed can aid in inferring dynamic motions along with parameters that adapt to both the tool and container. Our parameters are relative to the dimensions of the container, specifically the radius within which the motion is executed, as well as the dimensions of the tools used to execute the motion. It is crucial that the motion in the simulation does not exceed the predetermined dimensions, as this could have potentially serious consequences when applied in the real world.
\section{Mixing Task}

\begin{table}[H]
    \centering
    \begin{tabular}{|c|c|c|c|c|}
      \hline
      \textbf{Nr.} & \textbf{Ingredients} & \textbf{Tool} & \textbf{Container} & \textbf{Motion}  \\
      \hline
      1 & Liquid & WoodenSpoon & Salad Bowl & Whirlstorm Motion \\
      \hline
      2 & Dry & WoodenSpoon & Salad Bowl & Whirlstorm Motion \\
      \hline
      3 & Liquid + Dry & WoodenSpoon & Salad Bowl & Whirlstorm Motion \\
      \hline
      4 & Liquid + Wet & WoodenSpoon & Salad Bowl & Whirlstorm Motion \\
      \hline
      5 & Dry + Wet & WoodenSpoon & Salad Bowl & Whirlstorm Motion \\
      \hline
      6 & Wet & WoodenSpoon & Salad Bowl & VerticalCircular + Whirlstorm Motion \\
      \hline
    \end{tabular}
    \caption{Mixing Task}
    \label{tab:mixingtask}
  \end{table}

\begin{itemize}
  \item Inferred Parameters for \textbf{1-5}: 
   \begin{lstlisting}
    => radius_lower_bound_relative = 0.0, 
    radius_upper_bound_relative = 0.7
  \end{lstlisting}
  \item Inferred Parameters for \textbf{6}:
  \begin{lstlisting}
    => radius_lower_bound_relative = 0.0, 
    radius_upper_bound_relative = 0.7,
    vertical_increment = 0.1
  \end{lstlisting}
\end{itemize}

\begin{figure}[H]
  \includegraphics[scale=0.26]{Graphics/mixingtask_rahmen.jpg}
  \caption{Whirlstorm Motion \& Vertical Circular Motion + Whirlstorm Motion}
  \label{fig:mixingverb WikiHow}
\end{figure}

\section{Beating Task}

\section{Stirring Task}

\section{Whisking Task}

\section{Folding Task}

\chapter{Knowledge Graph Visualization Tool}
\label{chap:OWLViz}
Many existing knowledge graph visualizations are restricted to displaying certain types of relationships. 
Most commonly, a class hierarchy of an ontology is visualized for the user. Other graph visualization tools have built in features capable of 
displaying relationships between instantiated classes or \textit{NamedIndividuals}.
A few of these graph visualizers are mentioned in \nameref{chap:Related_work}.

What we didn't find is a graph visualizer capable of displaying complex class expressions, therefore, we decided to develop our own graph visualizer, 
which is capable of displaying class hierarchies with complex constructs.

This chapter will present the main features of our \textit{Knowledge Graph Visualization Tool}, where explain explain its capabilities and limitations.

\section{Main concept}
\label{sec:MainConceps}

We present you our three main concepts, which are incldued in the \textit{Knowledge Graph Visualization Tool}:
\begin{itemize}
    \item Visualization of Class Hierarchy: Our primary goal is to visualise a class hierarchy, where class relations are visible to the 
    user and visualise complex class expressions as a part of the corresponding class. This will be done as compact as possible. 
    Furthermore to differentiate between classes and class expressions we aim to make such distinction via different coloring. 
    Additionally, we aim to keep the amount of nodes and edges low, to reduce clutter in the graph, making exploring graph easier and 
    increase performance of visualizing such graphs.  
    \item \textit{Query Builder}: To clarify the relations between the queries, we implement a query builder that, given a class, can point to other classes through the relation, 
    which in turn can point to further classes through another relation, as long as the user desires or there are no further relations. 
    No \textit{SPARQL} knowledge is required query language, making it usable for the user. 
    The \textit{Query Builder} outputs a filtered graph with the selected triples and generates a \textit{SPARQL} query.
    \item \textit{Rule Based Inference}: Our thesis relies on running inference over \textit{mixing} 
    we implement an inference tool running reasoning over our ontology, to find motions, analogously to \nameref{subsection:MixingActionSWRL}.
    The result is visualized in a graph and a sequence of actions in natural language are generated, to highlight inferred knowledge.  
\end{itemize}

\section{Features}

\begin{figure}[H]
    \centering
    \includegraphics[scale=0.35]{Graphics/architecture_simplified.jpg}
    \label{fig:OWLViz_architecture}
    \caption{Architecture chart for the \textit{Knowledge Graph Visualization Tool} framework}
\end{figure}

\subsection{Graph Visualization}
In this section, we present a simple example of our framework, from processing the ontology to visualization in the web application. 
For this purpose, we create a small and simple ontology to explain the basic visualization of an ontolgy. 
The visualization library \textit{vis.js} \cite{visjs} has been used to create graph visualization in web-browsers.

\begin{figure}[H]
    \centering
    \includegraphics[scale=0.4]{Graphics/simple_ontology.jpg}
    \caption{Trivial example of an ontology}
\end{figure}
    
The ontology created for this purpose aims to provide a simple representation of various food products. 
It includes superclass categories such as \textit{Fruits, Vegetables, Meat, and Dairy Products}, as well as subclasses like \textit{Apple} and \textit{Orange}, 
which are subclasses of the \textit{Fruits} superclass. 
This visualization is a simple visualization of a class hierarchy of food products, where each node is connected via 
a directed edge labeled with \textit{subClassOf}.

\subsection{Graph Processing}
The core component of the graph visualizer is the visualization of a hierarchy of classes. In this hierarchy, two nodes are connected by a directed edge, 
where the source node is the child node and the target node is the parent node. All edges are labeled either as \textit{subClassOf} or \textit{equivalentClass}.

To create the visualization from the graph, the ontology is parsed, and a triple matching process is applied to identify 
all classes that use \textit{subClassOf} or \textit{equivalentClass} as predicates.

By default, all nodes are uniformly colored green. However, in the \textit{mixing} and \textit{FoodCutting} \cite{Kuempel2024} ontologies, 
additional colors are used to highlight differences.
\begin{figure}[H]
    \includegraphics[scale=0.5]{Graphics/OwlVisualizer/graphProcessing1.png}
    \centering
    \caption{Processed Class Hierarchy}
\end{figure}

The graph visualizer is also capable of visualizing complex class expressions. 
However, it is essential to understand why the parsed expressions should not be used directly, as illustrated in figure \ref{fig:graphProcessing2}. 
Blank nodes and meta information associated with these expressions are generally useless to the user. 
This type of visualization is not compact, and users unfamiliar with ontology parsing will find it confusing and uninformative. 
Additionally, visualizing each parsed node increases the number of nodes and edges, which decreases performance of creating the graph visualization. 
Therefore, a more compact visualization of complex restrictions is necessary to ensure clarity and maintain optimal performance.

\begin{figure}[H]
    \includegraphics[scale=0.5]{Graphics/OwlVisualizer/graphProcessing2.png}
    \centering
    \caption{Visualize parsed class restrictions}
    \label{fig:graphProcessing2}
\end{figure}

To achieve a compact representation of class expressions, we recursively traverse the parsed class expression, 
searching for essential information such as the relations used and their corresponding values. 
All blank nodes are implicitly skipped and not visualized. Additionally, meta information are excluded.

The visualizer can also handle nested expressions, as well as intersections and unions. Each \textbf{AND} node indicates that all connected elements belong to the intersection.
Similarly, each \textbf{OR} node indicates that all connected elements belong to the union.

In the figure \ref{fig:graphProcessing3}, the following expression is visualized:

CircularMotion \textit{subClassOf} (radius upper bound relative value 0.7) and (radius lower bound relative value 0.7)

\begin{figure}[H]
    \includegraphics[scale=0.3]{Graphics/OwlVisualizer/graphProcessing3.png}
    \centering
    \caption{Compact representation of class expression}
    \label{fig:graphProcessing3}
\end{figure}

\subsection{Load Ontology}

By pressing the \textit{Load Ontology} button, users can load a wide range of ontologies that have the \textit{.owl} file extension.
Upon selecting the ontology, the \textit{Knowledge Graph Visualization Tool} begins parsing the ontology to process the ontology. 
Once the processing is complete, a visual representation of the ontology is created, which can be explored by the users freely. 


\begin{figure}[H]
    \includegraphics[scale=0.35]{Graphics/OwlVisualizer/loadOntology1.png}
    \centering
    \caption{Loading Ontology}
    \label{fig:loadOntology1}
\end{figure}

Attempting to load any other file types results in an error message being displayed on the website. 
When a user tries to upload a file that does not have the \textit{.owl} extension, the \textit{Knowledge Graph Visualization Tool} notifies 
the user that the file extension is not supported by this tool, via an error message. 

\begin{figure}[H]
    \includegraphics[scale=0.5]{Graphics/OwlVisualizer/loadOntology2.png}
    \centering
    \caption{\textbf{Error}: Unsupported File Type}
    \label{fig:loadOntology2}
\end{figure}

Additionally, if an ontology can't be parsed, an error message is thrown that ontology couldn't be
loaded properly. Whenever this issue is encountered during the parsing of the selected ontology, it immediately stops and triggers an error notification. 
This message informs the user that there was a problem with loading the ontology, possibly due to a formatting error or an unsupported structure within the file. 
This error message informs the user of this particular issue; the \textit{Knowledge Graph Visualization Tool} doesn't attempt to deal with this problem.

\begin{figure}[H]
    \includegraphics[scale=0.5]{Graphics/OwlVisualizer/loadOntology3.png}
    \centering
    \caption{\textbf{Error}: Ontology Load Failure}
    \label{fig:loadOntology3}
\end{figure}

\subsection{Search Classes}
The \textit{Knowledge Graph Visualization Tool} includes a feature that suggests classes, which are visualized as nodes, 
for users to focus on, improving their ability to navigate and understand complex networks.

\begin{figure}[H]
    \includegraphics[scale=0.6]{Graphics/OwlVisualizer/searchClass1.png}
    \centering
    \caption{Search bar suggesting available classes}
\end{figure}

The key implementation for this feature is a search bar that suggests all visualized classes. 
Any nodes belonging to a class restriction are not suggested. 
The available suggestions are sorted in descending order, allowing users to easily search
for classes and explore the class hierarchy without assuming prior familiarity with the ontology.

\begin{figure}[H]
    \includegraphics[scale=0.25]{Graphics/OwlVisualizer/searchClass2.png}
    \centering
    \caption{Focus node}
\end{figure}

Users can press \textbf{Enter} inside the search bar or click the search button to immediately focus on the chosen class. 
Focusing on the class is achieved by zooming to the respective node and highlighting it. 
This allows users to quickly identify the corresponding node along with all of its parent and child nodes.

\subsection{Custom Inference - Mixing}
To understand which motion can be inferred using the \textit{mixing} ontology, we created
the \textit{Custom Inference} (see \nameref{chap:Data_representation}) feature. The user simply selects the task to 
execute and the ingredients to include in the specific task. 
\begin{figure}[H]
    \centering
    \includegraphics[scale=0.45]{Graphics/inference_user_input.png}
    \caption{Select Task and Ingredients.}
\end{figure}
The \textit{Custom Inference} feature generates a graph that contains assertional knowledge, such as the task and the ingredients. 
Furthermore, the inferred motion along with its parameters is visualized for the user in a single graph. 
Additionally, a sequence of steps is generated, which the robot can execute to successfully complete its task. 
To achieve this, the form data is processed, and a reasoning process is initiated to run inference over our \textit{SWRL} rules, extract all inferred knowledge, 
and create a visualization and a sequence of actions.

\begin{figure}[H]
    \centering
    \includegraphics[scale=0.5]{Graphics/new_inference_graph.png}
    \caption{Visualize Graph and Sequence of Actions.}
    \label{fig:graph_inferred}
\end{figure}
\subsection{Query Builder}
The \textit{Query Builder} is a feature that provides users with suggested triples, allowing them to explore the graph in an intuitive manner.
No \textit{SPARQL} knowledge is required. This feature generates a visualization of the selected triples and constructs a \textit{SPARQL} query, 
which can be modified to the users needs.

\paragraph{Triple Matching}
Triple Matching allows users to select a subject, predicate, and object to explore the graph. 
At the beginning of the triple matching process, all classes are available as subjects. 

\begin{figure}[H]
    \includegraphics[scale=0.4]{Graphics/OwlVisualizer/queryBuilder1.png}
    \centering
    \caption{View Possible Subjects for Selection}
\end{figure}

Only predicates that appear as edge labels connected to the subject in the graph are suggested. 

\begin{figure}[H]
    \includegraphics[scale=0.4]{Graphics/OwlVisualizer/queryBuilder2.png}
    \centering
    \caption{View Possible Predicates for Selection}
\end{figure}
Similarly, only objects connected to the subject via the selected edge are proposed. 

\begin{figure}[H]
    \includegraphics[scale=0.4]{Graphics/OwlVisualizer/queryBuilder3.png}
    \centering
    \caption{View Possible Objects for Selection}
\end{figure}
Once a triple is selected, the user can continue to explore the graph based on the previously chosen triples.

All complex class expressions are processed in a single triple matching step, whereas navigating through the class hierarchy occurs incrementally.

Whenever a triple is selected, the \textit{Query Builder} creates a tab with two items, one item visualizes the graph, the other item 
contains a \textit{SPARQL} query. 

\paragraph{View Graph}
In the \textit{View Graph}, users can inspect the graph based on their selection of triples. 
This view is a subgraph, which makes it easier to inspect the relationships between classes and classes with complex class expressions. 
Each class expression is displayed in a separate view, allowing users to focus on each class expression independently, which also reduces clutter in the graph. 
The class hierarchy has its own view, which can be incrementally updated with new classes to view hierarchical relationships. 
Separating the views ensures that users can navigate and explore different aspects of the graph, while also improving their comprehension of the ontology.

\begin{figure}[H]
    \includegraphics[scale=0.5]{Graphics/OwlVisualizer/queryBuilder4.png}
    \centering
    \caption{Graph View}
\end{figure}

\paragraph{SPARQL Query}
A \textit{SPARQL ASK} query is generated for each chosen triple. If the triple is querying the hierarchy, the \textit{SPARQL} query for querying the hierarchy is updated with the newest triples.
For each complex class expression, a separate query corresponds to the view of the subgraph.

This query is a queryable triple pattern that asks if this query pattern has a solution or not. 
Users can then independently modify and adapt this pattern to suit their specific needs. 
This feature will help users unfamiliar with querying ontologies by providing them with the triple pattern.
\begin{figure}[H]
    \includegraphics[width=0.5\textwidth]{Graphics/OwlVisualizer/queryBuilder5.png}
    \centering
    \caption{\textit{SPARQL} View}
\end{figure}

\section{Limiations}

\paragraph{Using vis.js and Visualization of NamedIndividuals}

The graph visualization library \textit{vis.js} suffers from poor performance when handling large graphs, acknowledged by its developers \cite{visjs}. 
This limitation primarily stems from the physics simulation used in its force-directed graph layout, 
which attempts to arrange nodes by separating and clustering them based on their connections. 
As the number of interconnected nodes and edges grows, this process becomes increasingly time-consuming, leading to decreased performance. 
While capable of displaying relationships and attributes of instance data effectively, 
\textit{vis.js} is not optimized for handling large ontologies with numerous instantiated classes. 
Due to these limitations, functionality for visualizing such ontologies has not been implemented.

\paragraph{Inference and Reasoning}

Although we have developed a custom reasoner tailored specifically for the \textit{mixing} ontology, 
the current implementation does not support reasoning over other ontologies apart from \textit{mixing}. 
This decision was made due to runtime concerns, particularly regarding the computational complexity of \textit{SWRL} rules. 
Loading an ABox ontology, characterized by numerous NamedIndividuals and a complex class hierarchy, into this framework could severely impact performance.
Such intensive computations could potentially disrupt the usability of other features within the \textit{Knowledge Graph Visualization Tool}.

\bibliography{literature}
\end{document}

