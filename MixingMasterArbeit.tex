\documentclass[	pdftex, 
								a4paper,
								11pt, DIV11, BCOR5mm,
								parskip,
								%openany,
								]{scrreprt}
\usepackage[utf8]{inputenc}
\usepackage[T1]{fontenc}

\usepackage[colorlinks=true,linkcolor=blue,citecolor=blue]{hyperref}
\usepackage{comment}
\usepackage[ngerman]{babel}
\usepackage[nonumberlist,toc,acronym]{glossaries}

\usepackage{fancyhdr}
\usepackage[section]{placeins}
\usepackage{graphicx}
\usepackage{listings}
\usepackage[german]{babelbib}

\usepackage{varwidth}
\usepackage{float}

\usepackage{epstopdf}
\usepackage{textpos}
\usepackage{changepage}
\usepackage{theorem}
\usepackage{caption}
\usepackage{framed}
\usepackage{booktabs}
\usepackage{tikz}
\usetikzlibrary{decorations.markings,arrows,positioning,trees,calc,fit,shapes}
\usepackage[noend]{algorithmic}
\usepackage{algorithm}
\usepackage{environ}
\usepackage{ulem}

%Kopfzeile hinzufuegen
\renewcommand{\sectionmark}[1]{\markboth{#1}{}} % set the \leftmark

\fancypagestyle{plain}{
    
    \fancyhead[R]{\leftmark}
    \fancyhead[L]{\myauthor}
    \renewcommand{\headrulewidth}{1pt}
    %\fancyfoot[R]{\thepage}
}

\pagestyle{fancy}

%Literaturverzeichnis hinzufuegen
\bibliographystyle{babplain}

% Glossar hinzufuegen
\makeglossaries
\newacronym{usda} {USDA}{US Department of Agriculture}
\newacronym{fdc}{FDC}{FoodData Central}
\newacronym{rest}{REST}{Representational state transfer}
\newacronym{dge}{DGE}{Deutsche Gesellschaft für Ernährung}
\newacronym{pal}{PAL-Value}{physical activity level-Value}
\newacronym{json}{JSON}{JavaScript Object Notation}
\newacronym{iri}{IRI}{International Resource Identifier}
\newacronym{http}{HTTP}{HyperText Transfer Protocol}
\newacronym{xml}{XML}{Extensible Markup Language}
\newacronym{cdno}{CDNO}{Compositional Dietary Nutrition Ontology}
\newacronym{obo}{OBO}{Open Biological and Biomedical Ontology}
\newacronym{sparql}{SPARQL}{SPARQL Protocol And RDF Query Language}
\newacronym{sql}{SQL}{Structured Query Language}
\newacronym{uri}{URI}{Uniform Resource Identifier}
\newacronym{gi}{GI}{Glykämischer Index}
\newacronym{bmi}{BMI}{Body-Mass-Index}



\begin{document}
\setcounter{page}{-2}
\pagestyle{empty}
\pagenumbering{none}
\newcommand{\firstreviewer}{Prof. Michael Beetz PhD}
\newcommand{\secondreviewer}{}
\newcommand{\supervisor}{Michaela Kümpel}
\newcommand{\thesistype}{Master Thesis}
\newcommand{\myauthor}{Naser Azizi, Sorin Arion}
\newcommand{\mymaintitle}{A Knowledgebase with which you can generate robot plan for multiple mixing actions}
\newcommand{\mysubtitle}{A nutrition ontology with web interface for customer specific dietary \newline recommendations} 
\newcommand{\mytitle}{\centering {\Huge \mymaintitle}\\[.3in] 
    {\Large \mysubtitle}}
\newcommand{\pdftitle}{\mymaintitle - \mysubtitle}
\newcommand{\formattedfronttitle}{{\mytitle}}
\newcommand{\formattedinnertitle}{{\mytitle}}

\begin{titlepage}
	\vspace*{-2.2cm}
	\begin{adjustwidth}{-0cm}{-2.3cm}
	\thispagestyle{empty}
        \begin{figure}
            \begin{minipage}{.4\linewidth}
	\begin{flushleft}
		\includegraphics[height=1.5cm]{Graphics/unilogo-transp.pdf}
	\end{flushleft}
    \end{minipage}
    \hspace{.2\linewidth}
            \begin{minipage}{.4\linewidth}
	\begin{flushright}
		\includegraphics[height=3.0cm]{Graphics/logo-ai-small.pdf}
	\end{flushright}
    \end{minipage}
\end{figure}
	  \vfill
	  %\scalebox{0.95}{
    	  %\begin{minipage}{1.2\textwidth}
    	  %{
    		%\formattedfronttitle 
    	  %}
    	  %\end{minipage}
	  %}
	\begin{center}
	  {\huge \mymaintitle}\\
	  
	  \vfill
	  {\Large \thesistype}\\[2.5ex]
	  {\Large\em \myauthor}
	  \vfill
	{  
      \renewcommand\arraystretch{1.5}
      \begin{tabular}{l@{\hspace{2em}}r@{\hspace{1ex}}p{7cm}}
     Pr\"ufer der \thesistype: & 1. & \firstreviewer\\
                                 & 2. & \secondreviewer\\
	Supervisor		    &   & \supervisor\\
   \end{tabular}
  }
	\end{center}
	\end{adjustwidth}

	\pagenumbering{arabic}
	\setcounter{page}{1}
	\pagestyle{plain}
	%\addcontentsline{toc}{chapter}{Eidesstaatliche Erkl\"arung}
	\chapter*{Eidesstattliche Erkl\"arung}
	
	Hiermit erkl\"aren wir, dass die vorliegende Arbeit selbstst\"andig angefertigt,
	nicht anderweitig zu Pr\"ufungszwecken vorgelegt und keine anderen als die
	angegebenen Hilfsmittel verwendet habe. S\"amtliche wissentlich verwendete
	Textausschnitte, Zitate oder Inhalte anderer Verfasser wurden ausdr\"ucklich als
	solche gekennzeichnet.
	
	Bremen, den \makeatletter\@date\makeatother
	
	\vspace*{1em}
	\rule{15em}{0.16667pt}\\
	\author{Naser Azizi, Sorin Arion}
	\makeatletter\@author\makeatother
	
	
	\normalem
	%Abstrakt
	%\addcontentsline{toc}{chapter}{Abstract}
	\chapter*{Introduction}

	\chapter*{Motivation}
	\begin{itemize}
		\item Households Robots
		\item the big dream
		\item robotic Knowledgebase can be compared to the brain
		\item by implementing the knowledge about mixing actions we reach one step further to be able to execute recipe actions.
		\item 
	\end{itemize}
	\chapter*{Related Work}

	\chapter*{How does the robot works / plans, parameters, used tools, etc}
	\begin{itemize}
		\item PyCram
		\item Pr2
		\item Ontology
		\item OWLReady
		\item SWRL
		\item 
	\end{itemize}
	\chapter*{Data acquisition}
	\begin{itemize}
		\item WikiHow Hyponyms
		\item Video analysis
		\item Structured the required parameters for the knowledge base
	\end{itemize}
	\chapter*{Data representation}
	\begin{itemize}
		\item Ingredients, Tools, etc.
		\item Different Tasks/Actions maps on different Motions
		\item Motion dependencies
		\item Fobidden Tools with Containers
		\item How is an action defined
		\item Examples of using the knowledge graph
	\end{itemize}
	\chapter*{Implementation}
	\begin{itemize}
		\item OWLReady
		\item SWRL
		\item Inferring parameters for the actual plan
	\end{itemize}
	\chapter*{Simulation}
	\begin{itemize}
		\item BulletWorld as Simulation platform -> how is the robot actually shown, which limitations exist
		\item Quering the knowledge base
		\item Inferring the parameters in the Simulation
		\item Task execution with multiple sequential Motions
		\item 
	\end{itemize}
	\chapter*{Evaluation}
	\begin{itemize}
		\item Evaluation of multiple motions multiple times
		\item Results of that
	\end{itemize}
	\chapter*{Simulation to RealWorld gap}
	\begin{itemize}
		\item Big Problem: Uncertaininty
		\item Perception Solution as first approach: RoboKudo
		\item Data acquisition: Blenderproc
		\item Model Training: YoloV8
		\item Results showing the real world Perception.
	\end{itemize}
	\chapter*{Summary / Fazit}
\end{titlepage}
avc
\end{document}

