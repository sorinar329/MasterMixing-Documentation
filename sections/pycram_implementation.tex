\section{Implementation in PyCram}

\subsection{Workflow}




\subsection{Asserted Knowledge}
Any robot capable of executing mixing motions need to have knowledge before executing a mixing task. 
The robot needs to know where the container and tools are located, for picking up and for navigation to Theoretically
respective places. It needs to know which ingredients have to be mixed, in simulation only their existence have to be known. 
It needs to know which task to execute, since we differentiate between different mixing tasks as discussed in this section. 


\subsection{Inferring parameters}
The robot doesn't know what kind of motion with fitting parameters it has to execute. Once all the assertional knowledge is available
to the robot, it is then capable of inferring motion and parameters through reasoning. 


\subsection{MixingActionSWRL}
This class has multiple functionalities to infer the relevant motion or compound motions to execute, with its associated 
parameters. This class needs asserted knowledge by the robot, which tool has to be used, which ingredients have to be mixed 
which task has to be executed. To enable inference on the Mixing knowledgegraph, we have to first initialize instances of 
all ingredients, task, tool and container. Additionally a motion instance is initialized using the top-level concept \textit{Motion}. 
SWRL rules are only applied onto instances, not on classes. 

Once the initialization has been completed, we run the inference, which either uses a reasoner like pellet or HermIt. 
Running the inference will cause a reclasification on the motion instance, i.e a motion becomes a circular motion or something else. 
Since we know which class it has been reclassified to, we can then extract all parameters from the respective class and set the motion to be executed as well.

The inferred knowledge is passed onto the MixingAction which executes the actual mixing motion. 






