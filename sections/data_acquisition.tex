\chapter*{Data acquisition}
	The first step in acquiring the needed data, was to acknowledge which task varations of mixing are actually important. So we had to analyze the word \textit{Mixing} and its hyponyms. 
	hyponyms are subordered words of a given word, for example one hyponym of mixing could be beating. By looking on different websites like "framenet, Wordnet" , we created a list of possible hyponyms, that could be taken under consideration.
	For all those hyponyms, we delegated a WikiHow extraction search which should show us, how many times one of these words occur, in the context of cooking.
	In the table below the results can be seen.
	
    \begin{table}[H]
        \centering
        \begin{tabular}{|c|c|}
          \hline
          \textbf{Hyponym} & \textbf{Occurance}  \\
          \hline
          Mixing & 5300 \\
          \hline
          Combining & 3700  \\
          \hline
          Stirring & 6810 \\
          \hline
          Folding & 970 \\
          \hline
          Merging & 7 \\
          \hline
          Beating & 1200 \\
          \hline
          Whipping & 1000 \\
          \hline
          Join & 57 \\
          \hline
          Coalesce & 1 \\
          \hline
          Amalgamate & 0 \\
          \hline
          Pair & 57 \\
          \hline
          Blending & 1400 \\
          \hline
    
        \end{tabular}
        \caption{Example Table}
        \label{tab:example}
      \end{table}
      

	From this table one can see that many hyponyms have a low occurance and will be therefore not considered.
	We consider the words: Mixing, Stiring, Beating, Combining, Whiping, Whisking and Folding. Though Blending achieved a moderate occurance, we decided upon not considering it, because it involves a lot of electronic mixing machines, which the robot could not handle.
	From now on we will refer to these words also as Tasks.
	After deciding upon which tasks we consider, we have to analyze every task execution and define a structure.
	We started by analyzing different videos on WikiHow that included at least one of our choosen task and divided the information by:
	
	Task, used Tools, used Container, Ingredients, Motion.

	\begin{itemize}
		\item Tasks: The tasks can differ in its execution, like the tasks Beating (Mixing can be gentle or vigorous, while beating involves a more forceful and rapid action) and Folding (To maintain a light and fluffy texture in a cake, folding is often used to add dry ingredients without overmixing), which will ultimately lead to a decision upon which Motion should be used in the specific use case.
		\item Tools: Different Tools are used for different tasks, which also has a correlation with the choosen Container.
		\item Container: Most Mixing Tasks use a mixing bowl for container, but also container like Pot and Pan when considering cooking in boiling water.
		\item Ingredients: Besides Tasks, the ingredients are the most important component of our knowledgebase regarding the decision making upon motions. We divide the Ingredients in 4 subcategories: Dry/Powder, Wet, Liquid and Solid.
		\item Motions: The motions used in the analyzed videos will help us upon creating a knowledgebase from which the robot should be able to infer which motions he will use in different situations.
	\end{itemize}

	In the following we want to present our analysis results regarding the videos from WikiHow, but also from well known cooks, like Jamie Oliver:


    \begin{table}[H]
        \centering
        \begin{tabular}{|c|c|c|p{4,5cm}|p{4,5cm}|}
            \hline
            \textbf{Task} & \textbf{Tool} & \textbf{Container} & \textbf{Ingredients} & \textbf{Description} \\
            \hline
            Beating & Whisk & Bowl & Egg yolk (Wet ingredient) & circular, swirling wildly around the bowl \\
            \hline
            Stirring & Whisk & Bowl & Beaten Egg Yolk (Wet), Parmesan(Powder) and Pepper (Powder) & Circular, from the inside to the outside. \\
            \hline
            Stirring & Tongs & Pan & Wet Mixture, Pasta (Solid) and Bacon (Solid) & Diving motions, circular but also straight lines. \\
            \hline
            Whisk & Fork & Bowl & Eggs (Wet) & Circular but also straight, wildly motion. \\
            \hline
            Mixing & Spatula & Pan & Eggs, melted butter (Wet) & Circular, from the inside to the outside, also diving. \\
            \hline
            Folding & Spatula & Pan & cooked eggs in melted butter (Wet) & Gently motion from the outisde to the inside straight, then moving about 90 degree before going to the inside again. \\
            \hline
            Mixing & Spoon & Cup & Dry yeast(Powder), Water (Liquid) & Circular \\
            \hline
            Mixing & Spoon & Bowl & Dry yeast, Water, Flour (Powder), Salt(Powder) & Whirlstorm-like motion. \\
            \hline 
        \end{tabular}
        \caption{Example Table}
        \label{tab:example}
      \end{table}
      

	From this videos we extracted informations about the executed motions. The following motions were defined:
	\begin{itemize}
		\item \textbf{Circular}: Moving the tool in a defined circular movement in the container, not changing the radius during execution
		\item \textbf{Whirlstorm}: Moving from the inside to the outside of the container with the tool, by circulating in an incremented radius.
		\item \textbf{Folding}: Gently motion, where you start from the outside, moving one straight line to the inner side, then picking the tool up and going to the initial state before moving the tool for about 90 degrees, then going back in a straight line to the inner side of the container again.
		\item \textbf{VerticalCircular}: Imagine a line which can be seen as the diameter of the container, from this line one can define certain regions on which you move the tool circular from side to side. This motion is used by the beating task.
		\item \textbf{CircularDivingToInner}: Starting from the outerside, moving the tool in the container arround its edge for about 270 degrees, before diving to the middle of the container. This motion is used by Tasks where is required to turn the Ingredients over.
	\end{itemize}

	By analyzing the videos we concluded that the motion decision is based on the task and Ingredients. Upon this thoughts we decided to design a decision tree regarding how the motions will be choosen by the robot.

\section*{Mixing}
\includegraphics[scale=0.5]{Graphics/MasterMixingParameter.pdf}